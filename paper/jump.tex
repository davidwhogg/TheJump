\documentclass[12pt, letterpaper]{article}

\begin{document}\thispagestyle{empty}\sloppy\sloppypar\raggedbottom\frenchspacing

\section*{The Jump: A fully probabilistic data-driven model for stellar spectra}

\noindent
DWH, others

\paragraph{Abstract:}
Physical models of stellar spectra have been extremely successful
at delivering parameters and chemical abundances of stars.
One data-driven model, \textsl{The Cannon}, has been even more
successful; being data-driven, it is constructed to fit stellar
spectra beautifully.
\textsl{The Cannon} has delivered the most precise chemical abundances ever for
stars, as validated by the consistency of the chemical abundances
it reports for stars in open clusters.
Here we generalize this model from a point estimate to a fully
probabilistic model.
\textsl{The Jump} has all of the flexibility and good design of
its predecessor, but fully marginalizes out uncertainties in its
internal parameters when it delivers stellar parameters and
chemical abundances (collectively ``labels'').
Acting on the \textsl{SDSS-IV} \textsl{APOGEE} data, the model
has 6.7 million parameters; the method relies on a custom Gibbs sampling
scheme, and an abundance of compute time.
Being fully probabilistic, it can propagate training-set label uncertainties,
spectral noise, and training-set objects with missing label or spectral data.
We show that \textsl{The Jump} provides labels that are better than those
from \textsl{The Cannon}, but also reasonable label uncertainties.
It improves its predecessor both because it marginalizes out internal
uncertainties, and because it doesn't require a training set on which
hard cuts have been made in signal-to-noise and label completeness.

\section{Introduction}

\section{Assumptions and method}

Here are the assumptions underlying \textsl{The Jump}:
\begin{itemize}\itemskip=0ex
\item foo
\item bar
\end{itemize}

\end{document}
