\documentclass[12pt, letterpaper]{article}

% text formatting
\setlength{\baselineskip}{2.4ex}
\thispagestyle{empty}\sloppy\sloppypar\raggedbottom\frenchspacing

% text macros
\newcommand{\project}[1]{\textsl{{#1}}}
\newcommand{\acronym}[1]{{\small{#1}}}
\newcommand{\sectionname}{Section}

% math macros
\newcommand{\Teff}{{T_{\mathrm{eff}}}}
\newcommand{\logg}{{\log g}}

\begin{document}

\section*{The Jump: A fully probabilistic data-driven \\
model for stellar spectra}

\noindent
DWH, others

\paragraph{Abstract:}
Physical models of stellar spectra have been extremely successful
at delivering parameters and chemical abundances of stars.
One data-driven model, \project{The Cannon}, has been even more
successful; being data-driven, it is constructed to fit stellar
spectra beautifully.
\project{The Cannon} has delivered the most precise chemical abundances ever for
stars, as validated by the consistency of the chemical abundances
it reports for stars in open clusters.
Here we generalize this model from a point estimate to a fully
probabilistic model.
\project{The Jump} has all of the flexibility and good design of
its predecessor, but fully marginalizes out uncertainties in its
internal parameters when it delivers stellar parameters and
chemical abundances (collectively ``labels'').
Acting on the \project{\acronym{SDSS-IV}} \project{\acronym{APOGEE2}} data, the model
has 6.7 million parameters; the method relies on a custom Gibbs sampling
scheme, and an abundance of compute time.
Being fully probabilistic, it can propagate training-set label uncertainties,
account for spectral noise, and use training-set objects with missing label or spectral data.
We show that \project{The Jump} provides labels that are better than those
from \project{The Cannon}, but also reasonable label uncertainties.
It improves its predecessor both because it marginalizes out internal
uncertainties, and because it doesn't require a training set on which
hard cuts have been made in signal-to-noise and label completeness.

\section{Introduction}

\section{Assumptions and method}

The assumptions underlying \project{The Jump} are in the following
itemized list. These assumptions are all questionable, but they form
the basis for the method. That is, the method is designed to be correct
in a hypothetical world in which these assumptions all hold. We will
return below in \sectionname~DWH to discuss what happens as these
assumptions are softened or fail.
\begin{itemize}\itemsep0ex
\item There are $N$ stars $n$, each of which has a
  continuum-normalized (or, really, just consistently normalized)
  spectral flux $y_{nm}$ observed at each of $M$ rest-frame
  wavelengths $\lambda_m$. If a star has been observed multiple times,
  we will assume (for now) that the multiple observations have been
  combined into a single effective observation.
\item For each pixel value $y_{nm}$ of each spectrum, there is a known
  noise variance $\sigma^2_{nm}$ that describes the noise contributing
  to this pixel. These noise contributions can be treated as Gaussian
  and independent, with correctly estimated variances.
\item The expectation value for the flux $y_{nm}$ for star $n$ at
  wavelength $\lambda_m$ is set by a small number $K$ of latent labels
  $x_{nk}$, which are the effective temperature $\Teff_n$,
  surface gravity $\logg_n$, and chemical abundances for star $n$.
  In detail, the expectation value for $y_{nm}$ varies quadratically
  with the labels. (It would be a valuable extension to put a more
  flexible model here, or a non-parametric model.)
\item For at least some stars $n$, at least some of the labels
  $x_{nk}$ have been noisily observed, with Gaussian noise with known
  (and correctly known) variances $\Sigma^2_{nk}$. We will
  differentiate the observed and latent labels by denoting the
  observed labels $\ell_{nk}$.
\item We have various prior expectations about the latent parameters.
  One is that the derivatives of the spectral expectations with
  respect to the true labels ought to be very sparse: Most labels are
  not involved at most wavelengths. Another is that the stellar
  parameters $\Teff, \logg$ have joint values that are plausible
  given models of stellar structure; that is, not all values of
  $\Teff, \logg$ are plausible.
\end{itemize}

\end{document}
